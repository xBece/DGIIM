\documentclass{article}

%% Created with wxMaxima 16.04.2

\setlength{\parskip}{\medskipamount}
\setlength{\parindent}{0pt}
\usepackage[utf8]{inputenc}
\DeclareUnicodeCharacter{00B5}{\ensuremath{\mu}}
\usepackage{graphicx}
\usepackage{color}
\usepackage{amsmath}
\usepackage{ifthen}
\newsavebox{\picturebox}
\newlength{\pictureboxwidth}
\newlength{\pictureboxheight}
\newcommand{\includeimage}[1]{
    \savebox{\picturebox}{\includegraphics{#1}}
    \settoheight{\pictureboxheight}{\usebox{\picturebox}}
    \settowidth{\pictureboxwidth}{\usebox{\picturebox}}
    \ifthenelse{\lengthtest{\pictureboxwidth > .95\linewidth}}
    {
        \includegraphics[width=.95\linewidth,height=.80\textheight,keepaspectratio]{#1}
    }
    {
        \ifthenelse{\lengthtest{\pictureboxheight>.80\textheight}}
        {
            \includegraphics[width=.95\linewidth,height=.80\textheight,keepaspectratio]{#1}
            
        }
        {
            \includegraphics{#1}
        }
    }
}
\newlength{\thislabelwidth}
\DeclareMathOperator{\abs}{abs}
\usepackage{animate} % This package is required because the wxMaxima configuration option
                      % "Export animations to TeX" was enabled when this file was generated.

\definecolor{labelcolor}{RGB}{100,0,0}

\begin{document}

\pagebreak{}
{\Huge {\sc Primer parcial MN - Grupo A}}
\setcounter{section}{0}
\setcounter{subsection}{0}
\setcounter{figure}{0}


\section{Calcula la norma 1 de la matriz A}



\noindent
%%%%%%%%%%%%%%%
%%% INPUT:
\begin{minipage}[t]{8ex}\color{red}\bf
(\%{}i1) 
\end{minipage}
\begin{minipage}[t]{\textwidth}\color{blue}
a:matrix([5,sqrt(33),2],[2,6,-3],[9.1,7.23,1.2]);
\end{minipage}
%%% OUTPUT:
\[\displaystyle
\tag{a}\label{a}
\begin{pmatrix}5 & \sqrt{33} & 2\\
2 & 6 & -3\\
9.1 & 7.23 & 1.2\end{pmatrix}\mbox{}
\]
%%%%%%%%%%%%%%%


\noindent
%%%%%%%%%%%%%%%
%%% INPUT:
\begin{minipage}[t]{8ex}\color{red}\bf
(\%{}i2) 
\end{minipage}
\begin{minipage}[t]{\textwidth}\color{blue}
n:matrix\_size(a)[1];
\end{minipage}
%%% OUTPUT:
\[\displaystyle
\tag{n}\label{n}
3\mbox{}
\]
%%%%%%%%%%%%%%%


\noindent
%%%%%%%%%%%%%%%
%%% INPUT:
\begin{minipage}[t]{8ex}\color{red}\bf
(\%{}i3) 
\end{minipage}
\begin{minipage}[t]{\textwidth}\color{blue}
norma\_1:apply(max,makelist(apply("+",abs(list\_matrix\_entries(col(a,i)))),i,1,n));
\end{minipage}
%%% OUTPUT:
\[\displaystyle
\mbox{}\\\mbox{rat: replaced -2.870000000000001 by -287/100 = -2.87}\mbox{}\]
\[\displaystyle
\]
\[\tag{norma\_ 1}\label{norma\_ 1}
\sqrt{33}+13.23\mbox{}
\]
%%%%%%%%%%%%%%%


\noindent
%%%%%%%%%%%%%%%
%%% INPUT:
\begin{minipage}[t]{8ex}\color{red}\bf
(\%{}i4) 
\end{minipage}
\begin{minipage}[t]{\textwidth}\color{blue}
float(\%);
\end{minipage}
%%% OUTPUT:
\[\displaystyle
\tag{\%{}o4}\label{o4} 
18.97456264653803\mbox{}
\]
%%%%%%%%%%%%%%%

\section{Halla la iteración 11 del método iterativo dado y calcula el
error absoluto con respecto a la solución del sistema x:[1,1,1]
usando la norma 1.}


\subsection{Datos}



\noindent
%%%%%%%%%%%%%%%
%%% INPUT:
\begin{minipage}[t]{8ex}\color{red}\bf
(\%{}i5) 
\end{minipage}
\begin{minipage}[t]{\textwidth}\color{blue}
A:matrix([1,2,4],[1,6,1],[7,5,1]);
\end{minipage}
%%% OUTPUT:
\[\displaystyle
\tag{A}\label{A}
\begin{pmatrix}1 & 2 & 4\\
1 & 6 & 1\\
7 & 5 & 1\end{pmatrix}\mbox{}
\]
%%%%%%%%%%%%%%%


\noindent
%%%%%%%%%%%%%%%
%%% INPUT:
\begin{minipage}[t]{8ex}\color{red}\bf
(\%{}i6) 
\end{minipage}
\begin{minipage}[t]{\textwidth}\color{blue}
B:entermatrix(3,3);
\end{minipage}
%%% OUTPUT:
\[\displaystyle
\mbox{}\\\mbox{Is the matrix  1. Diagonal  2. Symmetric  3. Antisymmetric  4. General}\mbox{}
\]
%%%%%%%%%%%%%%%


\noindent
%%%%%%%%%%%%%%%
%%% INPUT:
\begin{minipage}[t]{8ex}\color{red}\bf
Answer 1, 2, 3 or 4 : 
\end{minipage}
\begin{minipage}[t]{\textwidth}\color{blue}
4;
\end{minipage}


\noindent
%%%%%%%%%%%%%%%
%%% INPUT:
\begin{minipage}[t]{8ex}\color{red}\bf
Row 1 Column 1: 
\end{minipage}
\begin{minipage}[t]{\textwidth}\color{blue}
5/42;
\end{minipage}


\noindent
%%%%%%%%%%%%%%%
%%% INPUT:
\begin{minipage}[t]{8ex}\color{red}\bf
Row 1 Column 2: 
\end{minipage}
\begin{minipage}[t]{\textwidth}\color{blue}
0;
\end{minipage}


\noindent
%%%%%%%%%%%%%%%
%%% INPUT:
\begin{minipage}[t]{8ex}\color{red}\bf
Row 1 Column 3: 
\end{minipage}
\begin{minipage}[t]{\textwidth}\color{blue}
-1/42;
\end{minipage}


\noindent
%%%%%%%%%%%%%%%
%%% INPUT:
\begin{minipage}[t]{8ex}\color{red}\bf
Row 2 Column 1: 
\end{minipage}
\begin{minipage}[t]{\textwidth}\color{blue}
-1/6;
\end{minipage}


\noindent
%%%%%%%%%%%%%%%
%%% INPUT:
\begin{minipage}[t]{8ex}\color{red}\bf
Row 2 Column 2: 
\end{minipage}
\begin{minipage}[t]{\textwidth}\color{blue}
0;
\end{minipage}


\noindent
%%%%%%%%%%%%%%%
%%% INPUT:
\begin{minipage}[t]{8ex}\color{red}\bf
Row 2 Column 3: 
\end{minipage}
\begin{minipage}[t]{\textwidth}\color{blue}
-1/6;
\end{minipage}


\noindent
%%%%%%%%%%%%%%%
%%% INPUT:
\begin{minipage}[t]{8ex}\color{red}\bf
Row 3 Column 1: 
\end{minipage}
\begin{minipage}[t]{\textwidth}\color{blue}
3/56;
\end{minipage}


\noindent
%%%%%%%%%%%%%%%
%%% INPUT:
\begin{minipage}[t]{8ex}\color{red}\bf
Row 3 Column 2: 
\end{minipage}
\begin{minipage}[t]{\textwidth}\color{blue}
0;
\end{minipage}


\noindent
%%%%%%%%%%%%%%%
%%% INPUT:
\begin{minipage}[t]{8ex}\color{red}\bf
Row 3 Column 3: 
\end{minipage}
\begin{minipage}[t]{\textwidth}\color{blue}
5/56;
\end{minipage}
%%% OUTPUT:
\[\displaystyle
\mbox{}\\\mbox{Matrix entered.}\mbox{}\]
\[\displaystyle
\]
\[\tag{\%{}o6}\label{o6} 
\begin{pmatrix}\frac{5}{42} & 0 & -\frac{1}{42}\\
-\frac{1}{6} & 0 & -\frac{1}{6}\\
\frac{3}{56} & 0 & \frac{5}{56}\end{pmatrix}\mbox{}
\]
%%%%%%%%%%%%%%%


\noindent
%%%%%%%%%%%%%%%
%%% INPUT:
\begin{minipage}[t]{8ex}\color{red}\bf
(\%{}i7) 
\end{minipage}
\begin{minipage}[t]{\textwidth}\color{blue}
b:[7,8,13];
\end{minipage}
%%% OUTPUT:
\[\displaystyle
\tag{b}\label{b}
[7,8,13]\mbox{}
\]
%%%%%%%%%%%%%%%


\noindent
%%%%%%%%%%%%%%%
%%% INPUT:
\begin{minipage}[t]{8ex}\color{red}\bf
(\%{}i8) 
\end{minipage}
\begin{minipage}[t]{\textwidth}\color{blue}
c:[19/21,4/3,6/7];
\end{minipage}
%%% OUTPUT:
\[\displaystyle
\tag{c}\label{c}
[\frac{19}{21},\frac{4}{3},\frac{6}{7}]\mbox{}
\]
%%%%%%%%%%%%%%%


\noindent
%%%%%%%%%%%%%%%
%%% INPUT:
\begin{minipage}[t]{8ex}\color{red}\bf
(\%{}i9) 
\end{minipage}
\begin{minipage}[t]{\textwidth}\color{blue}
x:[1,1,1];
\end{minipage}
%%% OUTPUT:
\[\displaystyle
\tag{x}\label{x}
[1,1,1]\mbox{}
\]
%%%%%%%%%%%%%%%

\subsection{Calculo de la iteración 11 y del error absoluto}



\noindent
%%%%%%%%%%%%%%%
%%% INPUT:
\begin{minipage}[t]{8ex}\color{red}\bf
(\%{}i10) 
\end{minipage}
\begin{minipage}[t]{\textwidth}\color{blue}
X[0]:[1,-5.1,3];
\end{minipage}
%%% OUTPUT:
\[\displaystyle
\tag{\%{}o10}\label{o10} 
[1,-5.1,3]\mbox{}
\]
%%%%%%%%%%%%%%%


\noindent
%%%%%%%%%%%%%%%
%%% INPUT:
\begin{minipage}[t]{8ex}\color{red}\bf
(\%{}i11) 
\end{minipage}
\begin{minipage}[t]{\textwidth}\color{blue}
X[n]:=B.X[n-1]+c;
\end{minipage}
%%% OUTPUT:
\[\displaystyle
\tag{\%{}o11}\label{o11} 
{{X}_{n}}:=B\mathit{ . }{{X}_{n-1}}+c\mbox{}
\]
%%%%%%%%%%%%%%%


\noindent
%%%%%%%%%%%%%%%
%%% INPUT:
\begin{minipage}[t]{8ex}\color{red}\bf
(\%{}i12) 
\end{minipage}
\begin{minipage}[t]{\textwidth}\color{blue}
X[11];
\end{minipage}
%%% OUTPUT:
\[\displaystyle
\tag{\%{}o12}\label{o12} 
\begin{pmatrix}1.000000000006922\\
1.000000000090522\\
0.9999999999530083\end{pmatrix}\mbox{}
\]
%%%%%%%%%%%%%%%


\noindent
%%%%%%%%%%%%%%%
%%% INPUT:
\begin{minipage}[t]{8ex}\color{red}\bf
(\%{}i13) 
\end{minipage}
\begin{minipage}[t]{\textwidth}\color{blue}
error\_absoluto:apply("+",abs(list\_matrix\_entries(X[11])-x));
\end{minipage}
%%% OUTPUT:
\[\displaystyle
\tag{error\_ absoluto}\label{error\_ absoluto}
1.444360187008442{{10}^{-10}}\mbox{}
\]
%%%%%%%%%%%%%%%


\noindent
%%%%%%%%%%%%%%%
%%% INPUT:
\begin{minipage}[t]{8ex}\color{red}\bf
(\%{}i14) 
\end{minipage}
\begin{minipage}[t]{\textwidth}\color{blue}
float(\%);
\end{minipage}
%%% OUTPUT:
\[\displaystyle
\tag{\%{}o14}\label{o14} 
1.444360187008442{{10}^{-10}}\mbox{}
\]
%%%%%%%%%%%%%%%
                                                                                    Done by xBece 
\end{document}
